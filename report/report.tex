 \documentclass{CRPITStyle} 
%\usepackage{epsfig}   % Packages to use if you wish
%\usepackage{lscape}   %
\usepackage{hyperref}
\usepackage{listings}
\lstset{language=Java,
 captionpos=b,
 tabsize=3,
 frame=lines,
% keywordstyle=\color{blue},
 commentstyle=\color{darkgreen},
 stringstyle=\color{red},
 numbers=left,
 numberstyle=\tiny,
 numbersep=5pt,
 breaklines=true,
 showstringspaces=false,
 basicstyle=\footnotesize,
 emph={label}
}
\usepackage{xcolor}
\definecolor{dark-red}{rgb}{0.4,0.15,0.15}
\definecolor{dark-blue}{rgb}{0.15,0.15,0.4}
\definecolor{medium-blue}{rgb}{0,0,0.5}
\hypersetup{
    colorlinks, linkcolor={black},
    citecolor={dark-blue}, urlcolor={medium-blue}
}
\usepackage[authoryear]{natbib}
\renewcommand{\cite}{\citep}
\pagestyle{empty}
\thispagestyle{empty}

\begin{document}

\title{DBSCAN Visualisation Report}
\author{Jonathan Mackenzie}
\affiliation{$^1$ School of Computer Science, Engineering and Mathematics \\
Flinders University of South Australia, \\
PO Box 2100, Adelaide, South Australia 5001, \\
Email:~{\tt jonathan.mackenzie@flinders.edu.au}\\[.1in]}

\maketitle

\begin{abstract}
This report describes my Java implementation of the DBSCAN algorithm and associated visualisation. It describes the development process, the justification of data-structures used and explanation of features. All sources and data used for this project can be found here \url{https://github.com/JonnoFTW/DBSCAN-java}
\end{abstract}
\vspace{.1in}

\section{Introduction}
The Density-Based Spatial Clustering of Applications with Noise (DBSCAN) algorithm was originally proposed by \cite{Ester96adensity-based} and has applications where neighbours of a spatial object need to be found quickly such as graphical information systems (GIS) and collision detection in games and physics simulations. In this project, I use quadtrees as an indexing data structure for calculating clusters using the DBSCAN algorithm. Other data structures such as KD-tree were considered but Quadtree was chosen because of it's simplicity.

\section{Usage}
To run the program, simply run the executable jar attached. The application has 2 main areas in the GUI: the visualisation and the configuration. The visualisation displays the results. To configure the run of the algorithm, set minpts and epsilon or push the "calculate parameters" button to have them calculated for you. The load file button will prompt the user to select a file of whitespace separated integers, with 2 integers per line to indicate a point in space. The export clusters button will prompt the user to select an output location for clustering results. The recalculate clusters button can be used to recalculate the clustering after changing minpts (but not epsilon). This does not require recalculation of neighbours. Start will run the neighbouring and then clustering algorithms. The progress bar displays the progress. The text area displays a log of the applications workings.

\section{Application}
\subsection{Features}
The features of the application are:
\begin{enumerate}
\item Visualisation of clusters and noise
\item Exporting of clustering results
\item Progress bar of algorithm completion
\item User specified parameters for minpts and epsilon
\item Optional automatic calculation of minpts and epsilon parameters 
\item Iterative cluster calculation (useful when making small tweaks to minpts, avoiding multiple neighbouring calculations). Made possible by storing the neighbours of each point as a member of that point.
\end{enumerate}

\subsection{Internals}
Originally, the application was written in Python (with limited features compared to the Java implementation), but this proved too slow when calculating the neighbours of points using the method described . When developing the DBSCAN algorithm, 

Java was chosen as the development language because it allowed the rapid development of the GUI using a GUI builder and has sufficient runtime performace compared to python. Interally, the data-points are stored in a QuadTree; a data-structure that provides $O(\log n)$ neighbour searching, $O(n)$ memory and $O(n)$ generation. The na\"{\i}ve method of calculating the distance between all points, as described below requires $O(n^2)$ time, but with a simple optimisation we can achieve $O(\frac{n^2 -n}{2})$

\subsection{Quadtree}
The quadtree \citet{Samet:1984:QRH:356924.356930} is a hierarchical data structure that stores data by partitioning each level into four quadrants. Each node in the tree has four subregions, similar to the quadrants of a compass, where points are stored in relation to the parent node. This makes searching and  It can be thought of as a 2-dimensional binary tree, where insertion/searching compares on the x and y axes on each point. We refer to

To make DBSCAN iterative, we store each point's neighbours in a HashSet on that point, for a quadtree $D$ of $n$ points, this requires $O(k)$ where $k$ is the number of neighbours of each point and $O(n^2)$ memory in the worst case (when all points are neighbours of each other). Since a single range query has $O(\log n + k)$ complexity, the complexity for finding all neighbourings is $O(n \log n + k)$. The time is also increased when epsilon is increased (since you end up searching the entire tree in the worst case) so the complexity approaches $O(n^2)$ so care must be taken to balance the tree (ie. minimise its maximum depth) and keep epsilon at a reasonable value. 

I have achieved a balanced quadtree tree by sorting the array of elements by their $x,y$ coordinates ($O(n \log n)$ in Java), placing all points in a balanced binary tree which requires $O(n)$ time and performing a breadth first traversal over the binary tree and inserting in that order (there is probably a way to iterate over the array of points and get the same ordering and reducing the memory overhead, but I couldn't figure it out). This ensures that the maximum depth of the quadtree is minimised and the number of subdivisions at each level is maximised. To test this, read the main function in \verb+QuadTree.java+. The results of this test are listed below:

\begin{verbatim}
Epsilon: 2000
Runs: 10
a1.txt------------------
arbitrary Maxdepth :21
arbitrary avgdepth :12.684
sorted Maxdepth :16
sorted avgdepth :8.663
zorder Maxdepth :45
zorder avgdepth :23.533
arbitrary: 0.0901s 
sorted: 0.0493s 
zorder: 0.1091s 
a2.txt------------------
arbitrary Maxdepth :22
arbitrary avgdepth :13.277
sorted Maxdepth :16
sorted avgdepth :8.879
zorder Maxdepth :40
zorder avgdepth :21.285
arbitrary: 0.0816s 
sorted: 0.0530s 
zorder: 0.1558s 
a3.txt------------------
arbitrary Maxdepth :29
arbitrary avgdepth :14.420
sorted Maxdepth :17
sorted avgdepth :9.586
zorder Maxdepth :40
zorder avgdepth :20.490
arbitrary: 0.1114s 
sorted: 0.0801s 
zorder: 0.2213s 
pathological------------------
arbitrary Maxdepth :3000
arbitrary avgdepth :3000.000
sorted Maxdepth :12
sorted avgdepth :11.931
zorder Maxdepth :3000
zorder avgdepth :3000.000
arbitrary: 0.8146s 
sorted: 0.7591s 
zorder: 0.7713s 
\end{verbatim}

\section{Future}
I believe that the quadtree could be useful in creating another density based clustering algorithm because of its natural tendency to store data in clusters, assuming that the quadtree itself is balanced.


\bibliographystyle{agsm}  
\bibliography{references}
\end{document}

